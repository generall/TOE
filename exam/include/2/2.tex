\subsection{Анализ электрической цепи методом комплексных амплитуд}

Синусоидально изменяющийся ток (напряжения) удобно представлять в виде действительной части комплексного числа.

Согласно формуле Эйлера:

\begin{equation}
e^{j \alpha} = \cos(\alpha) + j \sin(\alpha)
\end{equation}

Таким образом:


\begin{equation}
 I_m \cos (\omega t + \phi ) = Re[I_m e^{\omega t + \phi}]
\end{equation}

Величину $I_m e^{\omega t + \phi}$ называют \textit{комплексной амплитудой } тока и обозначают $\dot I_m$

Сумма двух токов разной частоты есть геометрическая сумма векторов $\dot I_{m1} \& \dot I_{m2}$

Амплитуда результирующего тока определяется длинной суммарного вектора а начальная фаза - углом, образованным с действительной осью +1.

Дальнейший расчет сводится к расчету для линейных цепей постоянного тока с заменой соответствующих величин на их представление в виде комплексных амплитуд.

\pagebreak