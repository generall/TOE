\subsection{Преобразование Фурье. Частотный и фазовый спектры.}

Любую периодическую функцию $f(x)$ с периодом $2\pi$ , удовлетворяющую условиям Дирихле, можно разложить в ряд Фурье. Переменная величина $x$ связана со временем $t$ соотношением

$$
x = \omega t = 2\pi t /T
$$, где T - период ф-ции во времени.
Для такой функции ряд Фурье записывают так:
$$
f(x) = A_0 + A_{1}'\sin(x) + A_{2}'\sin(2x)+ ... + A_{1}''\cos(x) + A_{2}''\cos(2x) + ...
$$
Здесь
$$
A_0 = \frac{1}{2\pi}\int_0^{2\pi}f(x)dx
$$
$$
A'_k = \frac{1}{\pi}\int_0^{2\pi}f(x)\sin(kx)dx
$$
$$
A''_k = \frac{1}{\pi}\int_0^{2\pi}f(x)\cos(kx)dx
$$
пользуясь формулой сложения, получаем 
\begin{equation}
f(x) = A_0 + \sum_{k=1}^{\infty} A_k \sin(kx + \psi_k)
\end{equation}

Обозначим период функции $T$, основную частоту $\omega_0 = 2\pi/T$. Тогда ряд Фурье можно записать двумя способами:
$$
f(t) = A_0 + \sum_{k=1}^{\infty} A_k \sin(k\omega_0 t + \psi_k)
$$
или
$$
f(t) = A_0 + \sum_{k=1}^{\infty} A'_k \sin(k\omega_0 t) + A''_k\cos(k\omega_0 t)
$$
Здесь $A'_k = A_k\cos(\psi_k) ; A''_k = A_k\sin(\psi_k)$

Выражения для нахождения коэффициентов:

$$
A_0 = \frac{1}{T}\int_{-T/2}^{T/2} f(t) dt
$$

$$
A_k \cos(\psi_k) = \frac{2}{T} \int_{-T/2}^{T/2}f(t)\cos(k \omega_0 t) dt
$$

$$
A_k \sin(\psi_k) = \frac{2}{T} \int_{-T/2}^{T/2}f(t)\sin(k \omega_0 t) dt
$$

Представив sin в комплексной форме по Эйлеру, получаем:


\begin{equation}
f(t) = A_0 + \frac{1}{2j} \sum_{k = - \infty}^{k = \infty} \dot A_k e^{j k \omega_0 t} 
\end{equation}
,где
\begin{equation}
\dot A_{k} = A_k e^{j \psi k} = A_k \cos(\psi_k) + j A_k \sin(\psi_k) = A'_k + jA''_k
\end{equation}
Тогда 
\begin{equation}
\dot A_k = \frac{2j}{T} \int_{-T/2}^{T/2}f(t) e^{-j k \omega_o t} dt
\end{equation}
С учетом последней формулы,
\begin{equation}
f(t) = A_0 + \sum_{k = - \infty}^{k = \infty} e^{j k \omega_o t} \frac{1}{T} \int_{-T/2}^{T/2}f(t) e^{-j k \omega_o t} dt
\end{equation}

Под {\bfseries интегралом Фурье} понимают тригонометрический ряд, представляющий непериодическую функцию суммой бесконечно большого числа синусоид, амплитуды которых бесконечно малы, а аргументы соседних синусоид отличаются на бесконечно малые значения.

Формулу для интеграла Фурье получают из формулы для ряда Фурье(см. предыдущую формулу) предельным переходом, а именно $T \rightarrow \infty$
При этом на функцию накладывается обязательное условие сходимости $ \int_{-\infty}^{+\infty} f(t)dt $

Нетрудно заметить,что $A_0 \rightarrow 0$

Выполним преобразование интеграла, стоящего под знаком суммы:
$$
\frac{1}{T}\int_{-T/2}^{T/2} f(t) e^{-jk\omega_0 t} dt
$$
С этой целью положим $\omega = k\omega_0$ ($\omega$ есть текущая, изменяющаяся частота) В ряде Фурье разность двух смежных частот $\Delta\omega=\omega_0 =2\pi/T$ => $1/T = \Delta\omega/(2\pi)$ Т.к. T велико, можно заменить $\Delta\omega$ на $d\omega$, получаем:
\begin{equation}
 \frac{1}{T} \int_{-T/2}^{T/2}f(t) e^{-j k \omega_o t} dt =  \frac{d \omega}{2 \pi} \int_{-\infty}^{\infty}f(t) e^{-j \omega t} dt
\end{equation}

Функция 

\begin{equation}
S(j \omega) = \int_{-\infty}^{\infty}f(t) e^{-j \omega t} dt
\end{equation}

есть спектр амплитуд при непрерывном преобразовании фурье. Называется также {\bfseries Прямым преобразованием Фурье}

Т.к. функция $ S(j \omega) $ комплексная, то выделяют отдельно амплитудный спектр $ |S(j \omega)| $ и фазовый спектр $ \phi(j \omega)=arg(S(j \omega)) $
%arg - функция, возвращающая угол наклона комплексного вектора к +1 оси.

Тогда обратное преобразование Фурье будет представлять из себя интеграл по спектру.

\begin{equation}
f(t) = \frac{1}{2\pi}\int_{-\infty}^{\infty}S(j \omega)e^{j k \omega_o t} d \omega
\end{equation}
Последняя формула представляет собой запись {\bfseries интеграла Фурье}.

\pagebreak
