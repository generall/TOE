\subsection{Фильтры верхней частоты. Связь между параметрами деталей и полосой пропускания }


Фильтры представляю собой четырехполюсники, устанавливаемые между источником питания и нагрузкой, назначение которых заключается в том, что они пропускают без затухания или с малым затуханием токи одних частот, и не пропускают, или пропускают с большим затуханием токи других частот.

Диапазон частот, пропускаемых без затухания, называется полосой пропускания.

Фильтр верхних частот пропускает только высокие частоты и задерживает низкие.

Простейший электронный фильтр верхних частот состоит из последовательно соединённых конденсатора и резистора.

Фильтр может быть реализован, например, из последовательно соединенных катушки и резистора, где напряжение снимается с катушки.

Определим частоту среза.


\begin{equation}
\dot K = \frac{i\,\omega\,L}{R+i\,\omega \,L}
\end{equation}

\begin{equation}
|K| = \sqrt{\frac{{\omega}^{2}\,{L}^{2}\,{R}^{2}}{{\left( {R}^{2}+{\omega}^{2}\,{L}^{2}\right) }^{2}}+\frac{{\omega}^{4}\,{L}^{4}}{{\left( {R}^{2}+{\omega}^{2}\,{L}^{2}\right) }^{2}}}
\end{equation}


\begin{equation}
|K|=\frac{\left| \omega\right| \,\left| L\right| }{\sqrt{{R}^{2}+{\omega}^{2}\,{L}^{2}}} = \sqrt(2)
\end{equation}

\begin{equation}
\omega = \frac{R}{L}
\end{equation}


\pagebreak