\subsection{Теорема о эквивалентном генераторе}

По отношению к выводам выделенной ветви или отдельного элемента остальную часть сложной схемы можно заменить а)эквивалентным генератором напряжения с ЭДС Е , равной напряжению холостого хода на выводах выделенной ветви или элемента и с внутренним сопротивлением R0, равным входному сопротивлению схемы со стороны выделенной ветви или элемента; б)эквивалентным генератором тока с J, равным току короткого замыкания на выводах выделенной ветви или элемента, и с внутренней проводимостью G0, равной входной проводимости схемы со стороны выделенной ветви или элемента.


Вычисления:

\begin{itemize}
\item
Найти напряжение на разомкнутой ветви ab. Т.е. напряжение на бесконечном сопротивлении, подключенному к двухполюснику.
\item
Определить входное сопротивление двухполюсника при закороченных генераторах ЭДС и вырванных генераторах тока
\item
Вычислить ток по формуле
\begin{equation}
I = \frac{U_{ab}}{R+R_{in}}
\end{equation}

$R_{in}$ можно так же вычислить, измеряя ток КЗ, тогда $R_{in}=U_{ab}/I_{kz}$

\end{itemize}


\pagebreak