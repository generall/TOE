\subsection{Фильтры нижних частоты. Связь между параметрами деталей и полосой пропускания }


Пропускает низкие чистоты, не пропускает верхние.

Фильтр может быть реализован, например, из последовательно соединенных конденсатора и резистора, где напряжение снимается с конденсатора.

Определим частоту среза.


\begin{equation}
\dot K = -\frac{i}{\omega\,C\,\left( R-\frac{i}{\omega\,C}\right) }
\end{equation}

\begin{equation}
|K| = \sqrt{\frac{{R}^{2}}{{\omega}^{2}\,{C}^{2}\,{\left( {R}^{2}+\frac{1}{{\omega}^{2}\,{C}^{2}}\right) }^{2}}+\frac{1}{{\omega}^{4}\,{C}^{4}\,{\left( {R}^{2}+\frac{1}{{\omega}^{2}\,{C}^{2}}\right) }^{2}}}
\end{equation}


\begin{equation}
|K|=\frac{1}{\sqrt{{\omega}^{2}\,{C}^{2}\,{R}^{2}+1}} = \sqrt(2)
\end{equation}

\begin{equation}
\omega = \frac{1}{C R}
\end{equation}


\pagebreak