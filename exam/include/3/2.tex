\subsection{Характеристическое уравнение.}


После алгебраизации системы уравнений (см. \ref{sec:alg}) получаем систему уравнений относительно свободных токов.

Например:

$$
\left\{\begin{matrix}
i_{CB1}& - &i_{CB2}& - &i_{CB3} & = 0 \\ 
(L_1 p + R_1)i_{CB1}& + & R_2 i_{CB2} && &=0 \\ 
R_2 i_{CB2}& - &&&\frac{i_{CB3}}{C p}  & = 0 
\end{matrix}\right.
$$


Тогда решение этой системы методом Крамера представляет собой:


\begin{eqnarray}
i_{CB1} = \frac{\Delta_1}{\Delta}\\
i_{CB2} = \frac{\Delta_2}{\Delta}\\
i_{CB3} = \frac{\Delta_3}{\Delta}
\end{eqnarray}

Где $\Delta $ - определитель матрицы системы

$\Delta_1 ,  \Delta_2, \Delta_3 = 0$ т.к. в них присутствуют нулевые столбцы.
Следовательно:

\begin{eqnarray}
i_{CB1} = \frac{0}{\Delta}\\
i_{CB2} = \frac{0}{\Delta}\\
i_{CB3} = \frac{0}{\Delta}
\end{eqnarray}
Из физических соображений ясно, что токи не могут равняться нулю, по этому

\begin{equation}
\Delta = 0
\end{equation}
Что и является характеристическим уравнением, из которого определяется $p$

\pagebreak
