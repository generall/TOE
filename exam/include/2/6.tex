\subsection{Комплексная передаточная функция электрической цепи.}

Комплексная передаточная функция представляет собой отношение некоторого выходного значения, записанного в комплексной форме к некоторому входному значению также записанному в комплексной форме.

 Передаточная функция непрерывной системы представляет собой отношение преобразования Лапласа выходного сигнала к преобразованию Лапласа входного сигнала при нулевых начальных условиях.

Бывают комплексные передаточные функции по:

\begin{itemize}
\item {\bfseries напряжению}

$$
H_U(j\omega) = \frac{\dot U_{m2}}{\dot U_{m1}} = \frac{\dot U_2}{\dot U_1}
$$, где $U_{m1} , U_{m2}, U_1, U_2$ соответственно комплексные амплитудные и действующие значения напряжения на входе и на выходе.

\item {\bfseries току}
$$
H_I(j\omega) = \frac{\dot I_{m2}}{\dot I_{m1}} = \frac{\dot I_{2}}{\dot I_{1}}
$$, где $I_{m1} , I_{m2}, I_1, I_2$ соответственно комплексные амплитудные и действующие значения тока на входе и на выходе.

\item {\bfseries сопротивлению}
$$
H_Z(j\omega) = \frac{\dot U_{m2}}{\dot I_{m1}} = \frac{\dot U_2}{\dot I_1}
$$ ,где $I_{m1} , U_{m2}, I_1, U_2$ соответственно комплексные амплитудные и действующие значения тока и напряжения на входе и на выходе. 

\item {\bfseries проводимости}
$$
H_Y(j\omega) = \frac{\dot I_{m2}}{\dot U_{m1}} = \frac{\dot I_2}{\dot U_1}
$$ ,где $U_{m1} , I_{m2}, U_1, I_2$ соответственно комплексные амплитудные и действующие значения напряжения и тока на входе и на выходе.
\end{itemize}

Комплексные функции определяются на частоте $\omega$ сигнала воздействия и зависят только от параметров цепи.

При помощи каждой из этих функций можно получить полную информацию о нужном нам участке цепи. Самая удобная для получения и использования функция {\bfseries по напряжению} поэтому чаще всего используется именно она.

Т.к. комплексная передаточная функция представляет собой комплексное число, его можно представить в форме
$$
H(j\omega) = H_1(\omega) + jH_2(\omega)
$$
суммы действительной и мнимой части.
Модуль $|H(j\omega)|$ называется {\slshape амплитудно-частотной характеристикой цепи}, а аргумент $\phi(\omega) = argH(j\omega)$ комплексной передаточной функции называют {\slshape фазо-частотной характеристикой цепи} 

Нетрудно получить соотношения, связывающие АЧХ и ФЧХ с вещественными и мнимыми частями комплексной передаточной функции $H_1(\omega)$ и $H_2(\omega)$

$$
|H(\omega)| = \sqrt{H_1^2(\omega) + H_2^2(\omega)}
$$
$$
\phi(\omega) = \arctan(\frac{H_2(\omega)}{H_1(\omega)})
$$

АЧХ и ФЧХ являются наиболее фундаментальными понятиями теории цепей и широко используются на практике. Требования к АЧХ и ФЧХ различных устройств являются определяющими при проектировании любой аппаратуры связи.

К примеру, для $H_U$ коэффициента передачи по напряжению, АЧХ показывает отношение амптитуд сигнала(выходного к входному), а ФЧХ показывает разность фаз между {\bfseries входным и выходным } сигналами.
\pagebreak