\subsection{Активные и пассивные компоненты электрической цепи}

В любой схеме можно выделить одну ветвь, а остальную цепь условно изобразить в виде двухполюсника. Если в двухполюснике присутствует источника тока или ЭДС, то он называется \textit{активным} и обозначается буквой "А", если источников нет, то он называется \textit{пассивным} и обозначается буквой "П" или вообще не обозначается.

 К пассивным относятся элементы, в которых рассеивается (резисторы) или накапливается (катушка индуктивности и конденсаторы) энергия. 
 
Активные  - транзистор, диод$^{[what?]}$, радиолампа и т.п.


\pagebreak