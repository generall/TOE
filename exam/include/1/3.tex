\subsection{Согласованный режим работы электрической цепи}

Пусть дана цепь с источником ЭДС $E$, внутреннее сопротивление которого $R_{in}$, и сопротивлением нагрузки $R$.

Тогда мощность, выделяющаяся на нагрузке равна
\begin{equation}
P = I^2 R = \frac{E^2}{(R+R_{in})^2} R
\end{equation}

Определим сопротивление нагрузки, при котором мощность максимальна.

\begin{equation}
\frac{dP}{dR} = \frac{{U}^{2}}{{( R+R_{in}) }^{2}}-\frac{2\,R\,{U}^{2}}{{( R+R_{in}) }^{3}} = 0 \Longrightarrow R=R_{in}
\end{equation}

$R=R_{in}$ - точка экстремума, и т.к. при $R<R_{in} P'>0, R>R_{in} P'<0$, то $R=R_{in}$ - точка максимума.

\begin{equation}
P_{max} = \frac{E^2}{4 R_{in}}
\end{equation}

Такой режим работы называется согласованным, КПД $\eta =\frac{R}{R+R_{in}}  = 0.5$




\pagebreak


