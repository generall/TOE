\subsection{Метод суперпозиции}

Метод наложения — метод расчёта электрических цепей, основанный на предположении, что ток в каждой из ветвей электрической цепи при всех включённых генераторах, равен сумме токов в этой же ветви, полученных при включении каждого из генераторов по очереди и отключении остальных генераторов.

При этом генератор тока исключается из схемы путем радикальной дислокации, а генератор напряжения шунтируется.

Метод применим как для цепей постоянного, так и переменного тока.

Не применим для \textbf{нелинейных} цепей.



\pagebreak