\subsection{Прямое и обратное преобразование Лапласа}

В соответствие функции времени, называемой оригиналом, ставится функция переменной p, называемой изображением.

\begin{equation}
F(p) = \int_0^\infty e^{-p t} f(t) dt
\end{equation}

\begin{equation}
f(t) = \frac{1}{2 \pi i} \int_{\phi - i \infty}^{\phi - i \infty} e^{p t} F(p) dp
\end{equation}


Изображения некоторых функций:


\begin{itemize}
\item 
Струпенька

\begin{equation}
F(p)=\int_0^\infty e^{-p t} E dt =  \left[ \frac{E e^{-p t}}{p} \right]_0^\infty = \frac{E}{p}
\end{equation}


\item

Показательная функция:

\begin{equation}
F(p)=\int_0^\infty e^{-p t} e^{a t} dt = \int_0^\infty e^{-(p-a) t}  dt = \frac{1}{p-a}
\end{equation}

\item

$ sin \& cos$ 

\begin{equation}
F(p) = \int_0^\infty e^{-p t} \sin(a t) dt = \frac{a}{{p}^{2}+{a}^{2}}
\end{equation}

\begin{equation}
F(p) = \int_0^\infty e^{-p t} \cos(a t) dt = \frac{p}{{p}^{2}+{a}^{2}}
\end{equation}

\end{itemize}

Первая производная:


\begin{equation}
\int_0^\infty e^{-p t} \frac{df(t)}{dt} dt = \int_0^\infty e^{-p t} d [f(t)]
\end{equation}

Интегрируя по частям, получаем $\frac{df(t)}{dt} \risingdotseq p F(p) $

Изображение интеграла:
$$
\int_0^\infty e^{-p t} \left[ \int_0^t f(t) dt \right]  dt = - \frac{1}{p} \int_0^\infty  \left[ \int_0^t f(t) dt \right]  d[e^{-p t}]
$$
Интегрируя по частям, получаем $\int_0^t f(t) dt \risingdotseq \frac{F(p)}{p} $


Теорема смещения:

\begin{equation}
F(p) = \int_0^\infty e^{-p t} f(t - \tau) dt = e^{-p\tau} F(p)
\end{equation}

Теорема о изменении масштаба:

\begin{equation}
F(p) = \int_0^\infty e^{-p t} f(a t) dt = \frac{1}{a}F(\frac{p}{a})
\end{equation}

Преобразование Лапласа аддитивно как, собственно, и интеграл.




\pagebreak
