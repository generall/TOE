\subsection{Мощность в электрической цепи синусоидального тока. Баланс мощностей.}

Мгновенная мощность - произведение мгновенных значений тока и напряжения.
$p=u i$
Стоит выделить мгновенную мощность на
\begin{itemize}
\item Резистивном элементе
$$
p=ui = U_m I_m \sin^2(\omega t)
$$
или
$$
p = \frac{U_m I_m}{2} (1- \cos(2\omega t)
$$
\item Индуктивном элементе
$$
p=ui= \frac{U_m I_m}{2} \sin(2\omega t)
$$
\item Емкостном элементе
$$
p=ui = \frac{U_m I_m}{2} \sin(2\omega t)
$$
\end{itemize}

{\slshape Активная мощность} $P$  - среднее значение мгновенной мощности $p$ за период T

\begin{equation}
P=\frac{1}{T} \int_0^T pdt = \frac{1}{T} \int_0^T uidt
\end{equation}

Если ток $i = I_m\sin(\omega t)$, а напряжение на участке цепи $u = U_m\sin(\omega t + \phi)$ 
%из результатов интегрирования.
\begin{equation}
P = \frac{U_m I_m}{2}\cos(\phi) = UI\cos(\phi)  = I^2 R
\end{equation}
, где U \& I - действующие значения напряжения и тока, в $\sqrt{2}$ раз меньшие чем $I_m \& U_m$

Фактически, активная мощность представляет собой энергию, которая выделяется в единицу времени в виде теплоты на участке цепи с сопротивлением $R$. В самом деле,
$U\cos(\phi) = IR$, Следовательно
$$
P = UI\cos(\phi) = I^2R
$$

Реактивная мощность $Q = U I \sin(\phi)$ - мощность, обусловленная наличием реактивных элементов цепи. Разберемся, что же она из себя представляет.

Возьмём цепочку с последовательно соединенными R, L, C. Запишем выражение для мгновенного значения суммы энергий магнитного и электрического полей цепи:
$$
W = W_m + W_e = \frac{Li^2}{2} + \frac{Cu_C^2}{2} = \frac{LI^2}{2}(1-\cos(2\omega t)) + \frac{I^2}{2\omega^2 C}(1 + \cos(2\omega t))
$$

Видно, что $W$ имеет постоянную и переменную составляющие. 
$$
W = W_{me0} - w_{me}
$$

На создание постоянной составляющей была затрачена энергия в процессе становления данного периодического режима. 
Получим среднее значение энергии $w_{me}$, поступающей от источника за интервал времени от $-T/8$ до $+T/8$
$$
W_{sred} = \frac{4}{t}\int_{-T/8}^{+T/8} w_{me}dt = ... = \frac{2}{\pi \omega}I^2(X_L - X_C) =\frac{2}{\pi \omega} UI\sin(\phi) = \frac{2}{\pi \omega} Q 
$$

Таким образом, реактивная мощность Q пропорциональна среднему за четверть периода значению энергии, которая отдаётся источником питания на создание переменной составляющей электрического и магнитного поля индуктивной катушки и конденсатора.

За один период переменного тока эта энергия дважды отдаётся генератором в цепь и дважды он получает ее обратно, т.е. {\bfseries реактивная мощность способствует обмену энергии между генератором и приемником} 

Полная мощность
$S=UI$. Мощности также связаны зависимостью $S^2 = P^2 + Q^2$ - мощность, которую источник может отдать потребителю, если он представляет собой чисто активное сопротивление.


Баланс мощностей заключается в том, что полая мощность, выделяемая на источнике должна равняться полной мощности, потребляемой цепью.

\begin{equation}
\sum_{i=1}^n I_i^2 Z_i = \sum_{i=1}^k I_i U_i e^{j \phi_{kz}}
\end{equation} 


\pagebreak