\subsection{Четырехполюсники. Способы формирования описания поведения четырехполюсника. Система параметров}


Четырехполюсник - обобщенное понятие электрической цепи, рассматриваемой по отношению к четырем ее зажимам.

Бывают активные и пассивные, в зависимости от наличия источников тока или напряжения внутри.

Работа четырехполюсника характеризуется $U_1 U_2 I_1 I_2$;

Соотношения между токами и напряжениями на входе и выходе четырехполюсника могут быть записаны в виде шести систем уравнений.
(Все возможные перестановки), например:

$$
\left\{\begin{matrix}
\dot I_1  =&\dot Y_{11} \dot U_1  + & Y_{12} \dot U_2 \\ 
\dot I_2  =&\dot Y_{21} \dot U_1  + & Y_{22} \dot U_2
\end{matrix}\right.
$$


\pagebreak