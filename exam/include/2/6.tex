\subsection{Комплексная передаточная функция электрической цепи.}

Комплексная передаточная функция представляет собой отношение некоторого выходного значения, записанного в комплексной форме к некоторому входному значению также записанному в комплексной форме.

 Передаточная функция непрерывной системы представляет собой отношение преобразования Лапласа выходного сигнала к преобразованию Лапласа входного сигнала при нулевых начальных условиях. //Википедия

Бывают передаточные функции по:

\begin{itemize}
\item
Ток к напряжению
\item
Напряжение к току
\item
Ток к току
\item
Напряжение к напряжению
\end{itemize}

Т.к. функция есть комплексная величина, то ее модуль - отношение амплитуд выходного и входного сигналов, а $\phi = \arctan(\frac{\Im[\dot K]}{\Re[\dot K]})$ - сдвиг по фазу выходного сигнала относительно входного.

\pagebreak