\subsection{Переходная характеристика электрической цепи и ее связь с частотной характеристикой}
\subsection{Связь с частотной характеристикой}


Переходная характеристика - это реакция системы на входное воздействие в виде функции Хевисайда (единичное ступенчатое воздействие)

Пусть функция Хевисайда - $l(t)$, ее изобрадение $ \frac{1}{p}$ , а изображение переходной функции - $H(p)$, тогда

\begin{equation}
H(p) = \frac{K(p)}{p}
\end{equation} 

\begin{equation}
K(p) = p H(p)
\end{equation}


\begin{equation}
h(t) \risingdotseq  \frac{K(p)}{p}
\end{equation}

, где $K(p)$ - комплексная передаточная функция

Таким образом, зная $h(t)$ можно построить АЧХ и ФЧХ.

\begin{equation}
|K(p)| = \sqrt{ \Re[p H(p)]^2 + \Im[p H(p)]^2 } =\sqrt{ \Re[ p \mathcal{L}(h(t))]^2 + \Im[ p \mathcal{L}(h(t))]^2 } 
\end{equation}


\pagebreak
