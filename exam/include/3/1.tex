\subsection{Классический метод анализа переходных процессов в электрических цепях. Свободная и вынужденная составляющие решения уранений и их опредления}
\label{sec:alg} 


Приведем задачу о переходном процессе к решению линейного дифференциального уравнения с постоянными коэффициентами.

Из второго закона Кирхгофа:


\begin{equation}
R i + L \frac{di}{dt} + \frac{1}{C} + \int_0^t i dt = E
\end{equation}

Определение тока как функции времени будет являться решением данного уравнения.

Решение уравнения проводится разными способами, некоторые из них:

\begin{itemize}
\item
Классический
\item
Операторный
\item
Интеграл Дюамеля
\item
Метод пространственных состояний
\end{itemize}

Из курса математического анализа известно, что решением дифференциального уравнения является сумма частного неоднородного решения и общего неоднородного.

Т.е. для уравнения

\begin{equation}
R i + L \frac{di}{dt} = E
\end{equation}

$I = \frac{E}{R}$ - частное решение. Эта составляющая тока называется принужденной. Принужденная составляющая представляет собой составляющую, изменяющуюся с той же частотой, что и действующий в цепи ЭДС.

$I = -\frac{E}{R} e^{-\frac{R}{L}t}$ - общее однородное решение. Эта составляющая тока называется свободной. Затухает по экспоненциальному закону.

Сумма свободного и вынужденного составляющих определяет действующее значение токов или напряжений.

Решение однородного дифференциального уравнения записывается  в виде показательных функций $A e^{pt}$, таким образом уравнение для каждого свободного тока можно представить в виде $A e^{pt}=i_{CB}$. Постоянная A для каждого свободного тока, в общем случае, разная, а параметр $p$  одинаков для свободных токов ветвей.

Тогда составляя систему уравнений для свободных токов можно избавиться от операций дифференцирования и интегрирования заменив их на умножение на $p$ и деление на $p$ соответственно.

Классический метод заключается в поиске свободных токов

\begin{equation}
i = \sum_{i=1}^n A_i e^{p_i t}
\end{equation}

Где n - число корней характеристического уравнения. $p_i $ - корни.

Начальные значения токов и все их производные считаются известными (и равными нулю при нулевых начальных условиях).

Тогда дифференцируем уравнение 

\begin{equation}
i = \sum_{i=1}^n A_i e^{p_i t}
\end{equation}

столько раз, сколько потребуется для составления системы, составляем систему м решаем.



\pagebreak
