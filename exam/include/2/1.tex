\subsection{Закон Ома для реактивных компонентов электрической цепи}

Пусть дана цепь из последовательно соединенных резистора, конденсатора, катушки индуктивности и генератора синусоидального тока.

Тогда из 2-го закона Кирхгофа следует уравнение для мгновенных значений:

\begin{equation}
U_R+U_L+U_C=E
\end{equation}

\begin{equation}
i R + L \frac{d i}{d t}+\frac{1}{C} \int i d t = E
\end{equation}


Переходим к комплексному представлению


\begin{equation}
\dot I_m R + j \omega L \dot I_m +\dot I_m \frac{-j}{C \omega} = \dot E_m 
\end{equation}

\begin{equation}\label{eq:sinOhm}
\dot I_m (R + j \omega L - \frac{j}{C \omega}) = \dot E_m 
\end{equation}

Множитель $R + j \omega L - \frac{j}{C \omega}$ называют комплексным сопротивлением.

\begin{equation}
\dot Z = z e^{j \phi} = R + j \omega L - \frac{j}{C \omega}
\end{equation}


\begin{equation}
\dot Z = R + j X
\end{equation}
, где R - активное сопротивления, а X - реактивное сопротивление

\begin{equation}
X = \omega L - \frac{1}{\omega C}
\end{equation}

Уравнение ~(\ref{eq:sinOhm}) является законом Ома для цепи синусоидального тока

\pagebreak