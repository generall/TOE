\subsection{Мощность в электрической цепи синусоидального тока. Баланс мощностей.}

Мгновенная мощность - произведение мгновенных значений тока и напряжения.
$p=u i$

Активная мощность - среднее значение мгновенной мощности за период T

\begin{equation}
P=\frac{1}{T} \int_0^T pdt = \frac{1}{T} \int_0^T uidt
\end{equation}

Если ток на участке цепи отстает от напряжения на $\phi$, то

%из результатов интегрирования.
\begin{equation}
P = U \cos (\phi) I = I^2 R
\end{equation}
, где U \& I - действующие значения напряжения и тока, в $\sqrt{2}$ раз меньшие чем $I_m \& U_m$

Реактивная мощность $Q = U I \sin(\phi)$ - мощность, передаваемая между генератором и приемником энергии (катушками и конденсаторами). Получается из усреднения по времени энергии, передаваемой источником реактивным элементам.

Полная мощность $S^2 = P^2 + Q^2$ - мощность, которую источник может отдать потребителю, если он представляет собой чисто активное сопротивление.


Баланс мощностей заключается в том, что полая мощность, выделяемая на источнике должна равняться полной мощности, потребляемой цепью.

\begin{equation}
\sum_{i=1}^n I_i^2 Z_i = \sum_{i=1}^k I_i U_i e^{j \phi_{kz}}
\end{equation} 


\pagebreak