\subsection{Преобразование Фурье. Частотный и фазовый спектры.}


Постулируется, что любую периодическую функцию можно разложить в тригонометрический ряд, называемый рядом Фурье.


\begin{equation}
f(t) = A_0 + \sum_{k=1}^{\infty} A_k \sin(k \omega_0 t + \psi_k)
\end{equation}

Представив sin в комплексной форме по Эйлеру, получаем:


\begin{equation}
f(t) = A_0 + \frac{1}{2j} \sum_{k = - \infty}^{k = \infty} \dot A_k e^{j k \omega_0 t} 
\end{equation}


\begin{equation}
\dot A_{k} = A_k e^{j \psi k} = A_k \cos(\psi_k) + j A_k \sin(\psi_k)
\end{equation}

Выражения для нахождения коэффициентов:

\begin{equation}
A_k \cos(\psi_k) = \frac{2}{T} \int_{-T/2}^{T/2}f(t)\cos(k \omega_0 t) dt
\end{equation}

\begin{equation}
A_k \sin(\psi_k) = \frac{2}{T} \int_{-T/2}^{T/2}f(t)\sin(k \omega_0 t) dt
\end{equation}

\begin{equation}
\dot A_k = \frac{1}{2j} \int_{-T/2}^{T/2}f(t) e^{-j k \omega_o t} dt
\end{equation}

\begin{equation}
f(t) = A_0 + \sum_{k = - \infty}^{k = \infty} e^{j k \omega_o t} \frac{1}{T} \int_{-T/2}^{T/2}f(t) e^{-j k \omega_o t} dt
\end{equation}



Совершая предельный переход $\omega \rightarrow 0; T \rightarrow \infty$


\begin{equation}
 \frac{1}{T} \int_{-T/2}^{T/2}f(t) e^{-j k \omega_o t} dt =  \frac{d \omega}{2 \pi} \int_{-\infty}^{\infty}f(t) e^{-j \omega t} dt
\end{equation}

Функция 

\begin{equation}
S(j \omega) = \int_{-\infty}^{\infty}f(t) e^{-j \omega t} dt
\end{equation}

есть спектр амплитуд при непрерывном преобразовании фурье.

Т.к. функция $ S(j \omega) $ комплексная, то выделяют отдельно амплитудный спектр $ |S(j \omega)| $ и фазовый спектр $ \phi(j \omega)=arg(S(j \omega)) $
%arg - функция, возвращающая угол наклона комплексного вектора к +1 оси.

Тогда обратное преобразование Фурье будет представлять из себя интеграл по спектру.

\begin{equation}
f(t) = \frac{1}{2\pi}\int_{-\infty}^{\infty}S(j \omega)e^{j k \omega_o t} d \omega
\end{equation}

\pagebreak