\subsection{Особенности поведения резонансных контуров. Влияние сопротивления потерь на его свойства. }
\subsection{Добротность, полоса пропускания, прочие свойства}

\subsubsection{Параллельный контур}

При параллельном соединении конденсатора и индуктивности.

Резонанс наблюдается в том случае, когда сопротивление контура стремится к бесконечности.

\begin{equation}
\dot Z = \frac{\frac{L}{C}}{i(\omega L - \frac{1}{\omega C})}
\end{equation}

\begin{equation}
\omega_0 = \frac{1}{\sqrt{L C}}
\end{equation}

Соответственно, при приближении частоты внешнего воздействия к резонансной наблюдается резкое возрастание напряжения в контуре.

И резонанс токов. (т.к. напряжение одинаково)


Крутизна определяется добротностью контура Q.

Добротность определяется как отношение энергии, запасаемых в контуре, к энергии, рассеиваемой на внешней нагрузке.

Пусть есть цепь и в начальный момент времени конденсатор заряжен. Тогда ток через катушку:

\begin{equation}
I_L = \frac{U}{j \omega_0 L}
\end{equation}

Через внешнюю нагрузку:

\begin{equation}
I_R = \frac{U}{R}
\end{equation}

Из определения Q:

\begin{equation}
Q = \frac{P_L}{P_R} = \frac{R}{\omega_0 L} = R \sqrt{\frac{C}{L}}
\end{equation}

Полоса пропускания определяется диапазоном частот, при которых напряжение на контуре $\geq 0,707 E$.

Записываем выражение для \textbf{амплитуд} токов и напряжений (с квадратами), в полученном выражении делаем замену на добротность, получая выражение относительно относительной расстройки.

И для параллельного и для последовательного колебательных контуров справедлива следующая формула, определяющая границу полосы пропускания(не пропускания)

\begin{equation}
\omega_2 - \omega_1 = \frac{\omega_0}{Q}
\end{equation}

\subsubsection{Последовательный контур}

Образуется при последовательном соединении конденсатора и индуктивности.

Наблюдается резонанс напряжений.

Добротность определяется аналогичным образом и равняется 

\begin{equation}
Q = \frac{1}{R} \sqrt{\frac{L}{C}}
\end{equation}

Напряжение на контуре при резонансе $ U \rightarrow 0$.

Полоса не пропускания аналогична параллельному контуру, но $\leq 0,707 E$.

\begin{equation}
\omega_2 - \omega_1 = \frac{\omega_0}{Q}
\end{equation}

\pagebreak