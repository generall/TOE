\subsection{Метод суперпозиции}

Метод наложения — метод расчёта электрических цепей, основанный на предположении, что ток в каждой из ветвей электрической цепи при всех включённых генераторах, равен сумме токов в этой же ветви, полученных при включении каждого из генераторов по очереди и отключении остальных генераторов.

Обоснование:
Основываясь на МКТ, получив n уравнений,где n = количество независимых контуров, каждый контурный ток $I_{kk}$ можно получить при помощи правила Крамера, т.е.
$$ I_{kk} = \frac{\Delta_{k}}{\Delta} $$
, где $\Delta_{k}$ - определитель, полученный заменой k-ого столбца на столбец свободных членов. В нашем случае - [E].

Тогда определитель $\Delta_{k}$ равен 
$$ E_{11}A_{k1} + E_{22}A_{k2} + ... + E_{nn}A_{kn}  $$

,где $A_{km}$ - алгебраическое дополнение $\Delta$ k-ого столбца и m-ой строки.

Тогда ток $I_{kk}$ может быть представлен в виде
$$ I_{kk} =  E_{11}\frac{A_{k1}}{\Delta} + E_{22}\frac{A_{k2}}{\Delta} + ... + E_{nn}\frac{A_{kn}}{\Delta} $$

Постулируется, что всегда можно составить общее выражение для тока в k-й ветви сложной схемы, причем при составлении используются уравнения МКТ, а контуры выбраны так, чтобы k-ая ветвь входила только в один k-контур. В таком случае $I_{k} = I_{kk}$  тогда можно записать, что 
$$ I_{k} =  E_{11}\frac{A_{k1}}{\Delta} + E_{22}\frac{A_{k2}}{\Delta} + ... + E_{nn}\frac{A_{kn}}{\Delta} $$

В свою очередь, $ E_{ii} $ (контурная ЭДС) можно выразить через сумму ЭДС ветвей $ E_1 , E_2 ... E_n $. Сгруппировав коэффициенты при $E_i$, получаем 
$$ I_k = E_{1}g_{k1} + E_{2}g_{k2} + ... + E_{kn}g_{kn} $$

В частном случае, когда какая-либо ЭДС, например m-ая, входит только в 1 m-контур, $ g_{km} = \frac{\Delta_{km}}{\Delta} $   

При этом генератор тока исключается из схемы путем радикальной дислокации, а генератор напряжения шунтируется.

Метод применим как для цепей постоянного, так и переменного тока.

Неприменим для \textbf{нелинейных} цепей.



\pagebreak