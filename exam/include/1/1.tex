\subsection{Законы Ома и Кирхгофа для электрической цепи}

%тест
\subsubsection{Закон Ома для участка цепи, не содержащего источника ЭДС}

Устанавливает связь между током и напряжением на участке цепи

\begin{equation}
U_{ab} = I R
\end{equation}

\begin{equation}
I = \frac{\phi_a - \phi_b}{R}
\end{equation}

\subsubsection{Обобщенный закон Ома для участка цепи, содержащего ЭДС}

Позволяет определить ток на участке цепи с известной разностью потенциалов и ЭДС источника:

\begin{equation}
 I = \frac{(\phi_a - \phi_b) \pm E}{R}
\end{equation}

"$+$" при ЭДС, направленном по обходу, и "$-$" при ЭДС направленном против обхода.

\subsubsection{Первый закон Кирхгофа}

Алгебраическая сумма токов, подтекающих к узлу равна 0:

\begin{equation}
\sum I_i = 0
\end{equation}

Является следствием принципа непрерывности полного тока.

\begin{equation}
\oint_S \overrightarrow{\sigma} d \overrightarrow{S} = 0
\end{equation}
, где $ \overrightarrow{\sigma}$ - плотность тока. 

\subsubsection{Второй закон Кирхгофа}

\begin{itemize}
\item
	Алгебраическая сумма падений напряжений в любом замкнутом контуре равняется алгебраической сумме всех ЭДС в том же контуре.
	$\sum I R = \sum E $
\item
	Алгебраическая сумма всех напряжений вдоль любого замкнутого контура равна нулю
	$\sum U_{ij} = 0 $
\end{itemize}

Законы Кирхгофа справедливы для линейных  и нелинейных цепей при любом характере изменения напряжения во времени.

\pagebreak
