\subsection{Нелинейные электрические цепи}

Если цепь содержит элементы, ВАХ которых отлична от линейной функции, то такая цепь называется нелинейной.

С линейной частью нелинейной цепи можно осуществлять любые, справедливые для обычных линейных цепей, преобразования.

Статическая характеристика есть

\begin{equation}
R_{stat}=U/I
\end{equation}
при неизменных U,I;

Дифференциальная характеристика есть

\begin{equation}
R_{dif}=\frac{dU}{dI}
\end{equation}

Дифференциальная характеристика в общем случае не равна статической.

Для расчета цепей могут применяться методы двух узлов и эквивалентного генератора. Законы Кирхгофа так же справедливы.

%возможно хорошо бы еще добавить про последовательное и параллельное соединение.

\pagebreak