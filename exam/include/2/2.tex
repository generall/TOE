\subsection{Анализ электрической цепи методом комплексных амплитуд}
\label{sec:complex}
Синусоидально изменяющийся ток (напряжения) удобно представлять в виде действительной части комплексного числа.

Согласно формуле Эйлера:

\begin{equation}
e^{j \alpha} = \cos(\alpha) + j \sin(\alpha)
\end{equation}

Угол в этой формуле может быть любым. Предположим,что $\alpha$ изменяется по закону $\alpha = \omega t + \phi$

Тогда 
$$ e^{j \alpha} = \cos(\alpha) + j \sin(\alpha)
$$
Принимает вид
$$
e^{j \omega t + \phi} = \cos(\omega t + \phi) + j \sin(\omega t + \phi)
$$

Это ни что иное, как множитель при амплитудном значении какой-либо величины, изменяющейся во времени. К примеру, если мы возьмём не $e^{j \omega t + \phi}$, а $I_{m}e^{j \omega t + \phi}$, то поведение функции не изменится, просто ее амплитудное значение изменится.

Пусть имеется сигнал, изменяющийся по закону $i(t) = I_{m}\cos(\omega t + \phi)$
Можно показать, что 
\begin{equation}
 I_m \cos (\omega t + \phi ) = Re[I_m e^{\omega t + \phi}]
\end{equation}
Исторически сложилось, что за основу действительного сигнала обычно берут действительную часть мнимого числа. 
Но т.к. амплитуды  действительного значения сигнала и комплексного его представления совпадают, справедливо ввести величину $I_m e^{\omega t + \phi}$ , которую называют \textit{комплексной амплитудой } тока и обозначают $\dot I_m$

Метод комплексных амплитуд удобен тем, что можно легко подсчитать амплитудные значения на всех элементах цепи.

Сумма двух токов разной частоты есть геометрическая сумма векторов $\dot I_{m1} \& \dot I_{m2}$

Амплитуда результирующего тока определяется длинной суммарного вектора а начальная фаза - углом, образованным с действительной осью +1.

Дальнейший расчет сводится к расчету для линейных цепей постоянного тока с заменой соответствующих величин на их представление в виде комплексных амплитуд.

\begin{itemize}

\item Резистивный элемент
Пусть $i = I_{m}\sin(\omega t + \phi)$ согласно закону Ома, $u = iR = RI_{m}\sin(\omega t + \phi)$
Комплексная амплитуда тока может быть представлена как 
$$ \dot I_m = I_m e^{\omega t + \phi} $$
Комплексная амплитуда напряжения:
$$ \dot U_m = R\dot I_m = RI_m e^{\omega t + \phi}
$$
\item Индуктивный элемент
Пусть $i = I_{m}\sin(\omega t )$
Тогда $$ U_{L} = -e_{L} = U_{m} = -(-L\frac{di}{dt}) = \omega LI_{m}\sin(\omega t + \pi/2)$$
Тогда 
$$
\dot U_m = \omega L\dot I_m 
$$
Сопротивление катушки обозначается $X_L = \omega L$ Видно, что напряжение на катушке опережает ток на $\pi/2$
\item Ёмкостный элемент
Пусть $u = U_m\sin(\omega t)$
Тогда $q = Cu = CU_m\sin(\omega t)$
Следовательно
$$
i = \frac{dq}{dt} = \omega CU_m\cos(\omega t ) = \omega CU_m\sin(\omega t + \pi/2)
$$
Видно, что напряжение на конденсаторе отстаёт от тока, текущего через него по фазе на $\pi/2$ 
Емкостное сопротивление обозначается $X_C = \frac{1}{\omega C}$ , тогда $I_m = U_m/X_C$, а комплексная амплитуда напряжения
$$
\dot U_m = X_C \dot I_m
$$

\item Стоит заметить, что напряжения на конденсаторе и катушке действуют в противофазе.

\end{itemize}

\pagebreak