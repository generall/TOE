\subsection{Последовательность анализа цепи операторным методом.}
В общем случае порядок расчета переходных процессов операторным
методом следующий:

\begin{itemize}
\item
 Выбираются положительные направления токов в ветвях и записываются интегродифференциальные уравнения Кирхгофа для цепи после коммутации.
\item

С помощью преобразования Лапласа переходим к изображениям слагаемых в составленных уравнениях.

\begin{equation}
F(p) = \int_0^\infty e^{-p t} f(t) dt
\end{equation}

 Записываются те же уравнения для изображений с учетом независимых начальных условий в виде внутренних источников ЭДС.

\item
 Полученные в операторной форме алгебраические уравнения решаются относительно изображения искомой величины.
\item
 На основе полученного изображения находится оригинал искомой функции. В общем случае оригинал можно найти с помощью обратного преобразования Лапласа.

\begin{equation}
f(t) = \frac{1}{2 \pi i} \int_{\phi - i \infty}^{\phi - i \infty} e^{p t} F(p) dp
\end{equation}


Если $F(p)$ представлено в виде $F(p)=\frac{F_1(p)}{F_2(p)}=\frac{M(p)}{N(p)}$ , где $M(p)$ - многочлен степени m, $N(p)$
- многочлен степени n, n>m

, то

\begin{equation}
f(t) = \sum_{k=1}^n \frac{F_1(p_k)}{F_2'(p_k)} e^{p_k t}
\end{equation}

Где $p_k$ - корни $F_2(p) = 0$.

Если один из корней $p_k = 0$, то 



\begin{equation}
f(t) =\frac{F_1(0)}{F_3(0)} + \sum_{k=1}^n \frac{F_1(p_k)}{p F_3'(p_k)} e^{p_k t}
\end{equation}

Где $F_2(p) = p F_3(p)$


\end{itemize}


\pagebreak
